\documentclass[11pt,a4paper]{jsarticle}
%
\usepackage{listings,jlisting}
\usepackage{amsmath,amssymb}
\usepackage{bm}
\usepackage{ascmac}
\usepackage[dvipdfmx]{graphicx}
\usepackage{here}
%ここからソースコードの表示に関する設定
\lstset{
  basicstyle={\ttfamily},
  identifierstyle={\small},
  commentstyle={\smallitshape},
  keywordstyle={\small\bfseries},
  ndkeywordstyle={\small},
  stringstyle={\small\ttfamily},
  frame={tb},
  breaklines=true,
  columns=[l]{fullflexible},
  numbers=left,
  xrightmargin=0zw,
  xleftmargin=3zw,
  numberstyle={\scriptsize},
  stepnumber=1,
  numbersep=1zw,
  lineskip=-0.5ex
}
%ここまでソースコードの表示に関する設定
\begin{document}


\title{計数工学プログラミング演習最終レポート}
\author{計数工学科システム情報学コース3年\\03-190615\\工藤龍}
\maketitle

\section{課題内容}

疎行列の2乗を様々な手法で計算し,実行時間を測定した.
具体的には,行列の保持方法が二次元配列の場合と隣接リストの場合でアルゴリズムを分け,
また,計算方法の部分でもいくつかの種類を考えた.

\section{手法}

今回の実験に用いたのは,以下の4つのアルゴリズムである.
\begin{itemize}
\item dense ijk
\item dense ikj
\item sparce transpose
\item sparce access
\end{itemize}
以下,上記の四つの説明をする.
dense ijkは,二次元配列の形で行列を保持するアルゴリズムである.
次のような形で積を計算する.
\begin{lstlisting}
for (int i = 0; i < n; i++) {
    for (int j = 0; j < m; j++) {
        x = 0;
        for (int k = 0; k < n; k++) {
            x += A[i][k] * A[k][j];
        }
        M[i][j] = x;
    }
}
\end{lstlisting}

dense ikjは,同じように二次元配列の形で行列を保持するが,積の計算の順序がやや異なる.
具体的には以下のようになっている.
\begin{lstlisting}
for (int i = 0; i < n; i++) {
    for (int k = 0; k < n; k++) {
        for (int j = 0; j < m; j++) {
            M[i][j] += A[i][k] * A[k][j];
        }
    }
}
\end{lstlisting}

sparce transposeは,隣接リストの形で行列を保持するものである.
入力した行列と,その転置行列を考えることで計算する方針をとっている.

sparce accessは,transposeと同様に隣接リストで行列を保持するが,
access関数を用いることで,転置を考えずに直接積を計算している.

入力した行列は,matrixmarketの行列である.
実行時間の計測は,timeコマンドのuserの値を利用することとした.


\section{実験結果}

それぞれのアルゴリズムに関し,計算を5回繰り返して平均をとった.
結果は次の表のようになった.

\begin{table}[H]
\begin{center}
\caption{アルゴリズムごとの行列の大きさと計算時間[s]}
\begin{tabular}{lllll}
\hline
行列の大きさ& dense ijk & dense ikj & sparce transpose & sparce access \\ \hline 
39&0.007&0.007&0.007&0.008\\
49&0.007&0.007&0.007&0.009\\
118&0.010&0.008&0.011&0.029\\
274&0.040&0.020&0.017&0.203\\
443&0.165&0.045&0.018&0.548\\
1454&8.835&1.695&0.057&16.914\\
1612&16.052&2.317&0.066&22.695\\
1624&17.645&2.355&0.072&23.359\\
1723&22.589&2.754&0.127&27.605\\
5300&3291.228&95.127&1.385&879.153\\ \hline
\end{tabular}
\end{center}
\end{table}


その結果のグラフが次のものである.

\begin{figure}[H]
  \begin{center}
  \includegraphics[width=14cm]{../graph.png}
  \caption{行列の大きさと計算の所要時間の関係}
  \end{center}
\end{figure}

ここからわかることとしては,全体として計算の速さが
sparce transpose,dense ijk,dense ikj,
sparce access
の順だということである.
また,dense ikjとsparce accessは最後の行列でだけ
順番が逆転している.

\section{考察}

まず,sparce accessが一番遅かったことに関してであるが,
これは納得できることだ.何故ならば,access関数のアルゴリズムを
見てみると,一つの値を探すのに,最大でその値が存在するであろう
行全てを探索することになるからである.これは,
二次元配列で行列を保持している場合よりも大きくなる可能性がある.
一方,一行探索しなくても良い場合もあるので,行列の形によっては
二次元配列よりも早くなることもあるであろう.
sparce accessがdence ijkを最後の行列でだけ抜いているのは,
行列の形によってたまたまそうなったものと考えられる.

また,特徴的なのはdense ijkとdense ikjの大きな差である.
両者の違いは積の計算の順番だけであるのに,
このように顕著な差が現れるのは興味深い.
これは,以下のように理解できる.
まず,ijkの順であるとA[i][k]とA[k][j]の両方が
一番内側のループ内で毎回変わっていく.
一方,ikjの順であると,
A[k][j]は一番内側のループで毎回変わるが,
A[i][k]は変わらない.
計算に使用する値を毎回変更する必要がなくなるのが,
これだけの差が生まれた原因だと考えられる.

また,sparce transposeは,二次元配列ではなく
隣接リスト表現で値を保持し,かつ
転置してから計算するため,ベクトル同士の積として
計算することができる.ゆえに先のikjよりも
負担の少ない形で計算できるため,もっとも
実行時間の短いアルゴリズムとなっていると考えられる.

今後の展望としては,二次元配列のアルゴリズムについて
今回はikjとijkのみ扱ったが,ijk三つのならびに関して
6通りの場合が考えられる.それらの違いに関して実験し,
考察してみるということが考えられる.

\end{document}
